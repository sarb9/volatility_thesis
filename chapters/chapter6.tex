\chapter{جمع‌بندی، نتیجه‌گیری و پیشنهادات}
%%%%%%%%%%%%%%%%%%%%%%%%%%%%%%%%%%%%%%%%%%%
\section{جمع‌بندی و نتیجه‌گیری}
در این پروژه، در ابتدا مفهوم مثال مخرب در حوزه‌های بینایی ماشین و پردازش زبان طبیعی را مورد بررسی قرار دادیم. سپس روش‌های تولید مثال‌های مخرب و همچنین دفاع در برابر آن‌ها را توضیح دادیم. شرح دادیم که حملات مخرب در متن در سطوح مختلف کاراکتر، واژه، جمله و چندسطحی تعریف می‌شوند و راهکارهای دفاعی ارائه ‌شده در کارهای مرتبط در سال‌های اخیر بر دفاع در برابر یک شیوه حمله خاص تمرکز داشته‌اند. گفتیم که تمرکز ما در این پروژه بر دفاع برابر حملات مخرب متنی در سطح واژه به صورت غلط املایی است و چندی از مقالات مرتبط در زمینه دفاع در برابر حملات در سطح واژه را که از روش‌هایی مانند خوشه‌بندی و شبکه‌ عصبی بازگشتی استفاده کرده بودند معرفی کردیم. در قسمت روش پیشنهادی، ابتدا روش‌های مختلف بازنمایی متن را معرفی کردیم و به شرح مدل‌های مبتنی بر انتقال‌دهنده‌ها که اخیرا در مسائل مختلف حوزه پردازش زبان استفاده شده‌اند پرداختیم. توضیح دادیم که چگونه می‌توان با استفاده از ویژگی پوشش واژ‌ه‌ها در مدل برت یک معیار برای امتیازدهی به معناداری کلمات در یک جمله در نظر گرفت و بر اساس این معیار تشخیص داد کدام واژه در یک جمله احتمالا یک واژه مخرب است. سپس برای تصحیح واژگانی که مخرب شناخته‌ شده‌اند، از لغت‌نامه گلاو استفاده کردیم تا با بررسی تمام همسایگان کاراکتری واژه مخرب متوجه شویم کدام یک از این همسایگان کلمه‌ای بامعنی است و در نهایت واژه‌ای که بیشترین امتیاز معناداری را داشته باشد به عنوان جایگزین کلمه مخرب معرفی می‌کردیم. در قسمت پیاده‌سازی نیز کتابخانه‌های تکست‌اتک و هاگینگ‌فیس و چگونگی استفاده ما از آن‌ها برای تولید مثال‌های مخرب و مدل‌های از پیش‌ آموزش‌دیده را توضیح دادیم. در نهایت نتایج مدل پیشنهادی‌مان را بر روی سه دادگان یلپ پولاریتی، روتن تومیتوز و ای‌جی نیوز ارائه کردیم و نشان دادیم در تمام نتایج عملکرد مدل‌مان از روش نیمه‌کاراکتری بهتر بوده است.

\section{پیشنهادات}
همانطور که در بخش مقدماتی پروژه اشاره شد، روش‌های متنوعی برای تولید حملات مخرب در متن تا کنون ارائه شده است. روشی که با استفاده از پوشش واژه‌ها در برت ارائه دادیم قابلیت تشخیص حملات در سطح کلمه چه به صورت غلط املایی و چه به صورت مترادف را دارد. بنابراین یکی از پیشنهادات ما در این حوزه طراحی مدلی برای تصحیح کلمات مخرب تشخیص‌ داده‌شده‌ای است که به صورت مترادف کلمات اصلی به کار گرفته شده‌اند. همچنین با توجه به بار محاسباتی بالای محاسبه امتیاز جملات با استفاده از ویژگی پوشش واژه‌ها، ارائه راهکاری برای کمینه‌کردن این امر در هنگام پردازش داده‌ها می‌تواند کمک شایانی به تسریع و بهبود کیفیت روند پروژه نماید.