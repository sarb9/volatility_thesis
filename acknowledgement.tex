%%%%%%%%%%%%%%%%%%%%%%%%%%%%%%%%%%%%
\newpage\thispagestyle{empty}
% سپاس‌گزاری
{\nastaliq
سپاس‌گزاری
}
\\[2cm]
اکنون که مراحل پژوهش، تدوین و نگارش پایان نامه به پایان رسیده است، از مادر و پدر عزیزتر از جانم متشکرم که درطول زندگی و دوران تحصیل همراه و مشوقم بوده‌اند و با ایثار و از خودگذشتگی و تحمل زحمات، مرا در این راه یاری نمودند. از سرکار خانم دکتر مزلقانی که به عنوان استاد راهنما با سعه صدر در مسیر این پژوهش، همواره راهنما و راهگشای اینجانب بوده‌اند تقدیر و تشکر می‌نمایم.















% با استفاده از دستور زیر، امضای شما، به طور خودکار، درج می‌شود.
\signature








%%%%%%%%%%%%%%%%%%%%%%%%%%%%%%%%%%%%%%%%%
%%%%%%%%%%%%%%%%%%%%%%%%%%%%%%%%%کدهای زیر را تغییر ندهید.
\newpage\clearpage

\pagestyle{style2}

\vspace*{-1cm}
\section*{\centering چکیده}
%\addcontentsline{toc}{chapter}{چکیده}
\vspace*{.5cm}
%\ffa-abstract
هدف از انجام این پروژه پیش‌بینی نوسان قیمت در بازار رمزارزها با استفاده از اطلاعات دفتر سفارشات و ویژگی‌های استخراج شده از آن و شبکه‌هاي عصبی بازگشتی است. پیشبینی نوسان در بازارهاي مالی به طور کلی از اهمیت زیادي برخوردار است. به صورت سنتی تنها از اطلاعات و تاریخچه‌ی قیمت براي محاسبه‌ي نوسان در مدل‌هاي آماري استفاده شده‌است. در حالی که استفاده از اطلاعات موجود در دفتر سفارشات که شامل تمامی سفارشات موجود و در جریان یک بازار است می‌تواند دقت این مدل‌ها را افزایش دهد. اما به دلیل حجم زیاد دفتر سفارشات،‌ در گذشته کمتر از آن‌ در جهت پیش‌بینی نوسان استفاده شده است.\\
در این پروژه با واکشی و پیش‌پردازش اطلاعات دفتر سفارشات و سپس با استفاده از مدل‌هاي یادگیري شبکه‌ی عصبی بازگشتی، سعی در مدل کردن نوسان در بازارهاي رمزارزها را داریم. نتایج به دست آمده نشان‌دهنده‌ی برتری مدل‌های یادگیری شبکه‌ی عصبی بازگشتی در مقایسه با دیگر معماری‌های شبکه‌های عصبی در پیش‌بینی نوسان است.
\vspace*{2cm}


{\noindent\large\textbf{واژه‌های کلیدی:}}\par
\vspace*{.5cm}
پیش‌بینی سری‌‌های‌ زمانی، نوسان قیمت،‌ شبکه‌های عصبی، یادگیری عمیق، ‌شبکه‌های عصبی بازگشتی، دفتر ثبت سفارشات